% \long allows the input to span blank lines
\long\def\ignore#1{}

% Ensure that we are being run with XeTeX
\ifdefined\XeTeXversion\else
\errmessage{%
crash^^0a
============================================================^^0a
ERROR: This document requires XeTeX^^0a
============================================================}%
\fi

\def\shell#1{%
\ifnum\shellescape=1
\immediate\write18{#1}%
\else
\errmessage{%
crash^^0a
============================================================^^0a
ERROR: This file requires the -shell-escape flag^^0a
============================================================}%
\fi
}%

% Plain TeX chooses to leave the definition of \\ to the user. The most
% sensible definition for \\ is as a literal backslash character, so define it
% as such.

\def\\{\char`\\}%

% The null character ^^00 can be very useful as a sentinel marker when
% processing token lists. By default we cannot use ^^00 because Plain TeX
% assigns it a catcode of 9, which causes it to be completely ignored in all
% cases. Here we give it a catcode of 12 instead, making it usable.

\catcode`\^^00=12

% The @ symbol is intended be used as a prefix for internal macros. For
% example, \@rawmunch is prefixed with @ because it is not intended to be
% called by the user. At the bottom of this source file the catcode for @ is
% set back to 12.

\catcode`\@=11

% TeX usually eats and ignores newlines in the input file, making it hard to
% format message with \errmessage. This setting allows us to type ^^0a to get
% newlines where we want them.

\newlinechar="0A

% The \prot macro prevents the redefinition of names. One can do
% \prot\def\foo{text here}, and a crash will result if \foo is already
% defined. One may also use constructs like \prot\chardef\foo,
% \prot\countdef\foo, and other such things.

\def\prot#1#2{%
\ifdefined#2
\errmessage{%
crash^^0a
============================================================^^0a
ERROR: The control sequence \string#2 is already defined^^0a
============================================================}%
\fi
#1#2%
}%

% The standard page size is 8.5in by 11in, with an hsize and vsize of 6.5in
% and 9in. But this is too big for 10pt fonts. If we scale each of these
% values down by 10/12 then 10pt is made to look like 12pt font, at least in
% terms of its ratio to the page.

\pdfpagewidth=7.0833in
\pdfpageheight=9.1667in
\hsize=5.4167in
\vsize=7.5000in
\hoffset=-0.1667in
\voffset=-0.1667in

% Prevent widows and clubs at all costs.
\widowpenalty=10000
\clubpenalty=10000

% +----------------------+
% | TeX Assembler Macros |
% +----------------------+

% ===[ Stack pointers ]===

% These macros use the letters c, d, s, m, b, and t to mean count, dimen,
% skip, muskip, box, and toks. Recall that "100 is hex 0x100, and not decimal
% 100. Plain TeX was written to use registers 0x00 through 0xFF, so using
% 0x100 and up leaves us clear of any interference.

% Recall that \countdef\spc="100 defines \spc as another name for register
% 0x100, while \spc="105 puts the value 0x105 into that register.

\countdef\spc = "100 % counter stack pointer
\countdef\spd = "101 % dimen   stack pointer
\countdef\sps = "102 % skip    stack pointer
\countdef\spm = "103 % muskip  stack pointer
\countdef\spt = "104 % toks    stack pointer
\countdef\spb = "105 % box     stack pointer
\spc="105

\def\cref#1{\count\numexpr\spc - #1\relax}%
\def\dref#1{\dimen\numexpr\spd - #1\relax}%
\def\sref#1{\skip\numexpr\sps - #1\relax}%
\def\mref#1{\muskip\numexpr\spm - #1\relax}%
\def\tref#1{\toks\numexpr\spt - #1\relax}%
\def\bref#1{\numexpr\spb - #1\relax}%

\def\pushc#1{\global\cref{-1}=#1\global\advance\spc by 1\relax}%
\def\pushd#1{\global\dref{-1}=#1\global\advance\spd by 1\relax}%
\def\pushs#1{\global\sref{-1}=#1\global\advance\sps by 1\relax}%
\def\pushm#1{\global\mref{-1}=#1\global\advance\spm by 1\relax}%
\def\pusht#1{\global\tref{-1}={#1}\relax\global\advance\spt by 1\relax}%
\def\pushb#1{%
\global\setbox\numexpr\spb + 1\relax=#1\global\advance\spb by 1\relax}%

\def\popc#1{\global\advance\spc by -#1\relax}%
\def\popd#1{\global\advance\spd by -#1\relax}%
\def\pops#1{\global\advance\sps by -#1\relax}%
\def\popm#1{\global\advance\spm by -#1\relax}%
\def\popt#1{\global\advance\spt by -#1\relax}%
\def\popb#1{\global\advance\spb by -#1\relax}%

\def\flusht{\the\tref0\popt1\relax}%
\def\flushc{\number\cref0\popc1\relax}%

% Declares a new var, \var\count\foo

\def\var#1#2{%
\ifx#1\count  \advance \spc by 1 \prot\countdef#2  = \spc \fi
\ifx#1\dimen  \advance \spd by 1 \prot\dimendef#2  = \spd \fi
\ifx#1\skip   \advance \sps by 1 \prot\skipdef#2   = \sps \fi
\ifx#1\muskip \advance \spm by 1 \prot\muskipdef#2 = \spm \fi
\ifx#1\toks   \advance \spt by 1 \prot\toksdef#2   = \spt \fi
\ifx#1\box    \advance \spb by 1 \prot\chardef#2   = \spb \fi
}%

% Declares and sets a new var, \var\count\foo=7

\def\ivar#1#2{%
\var#1#2%
\ifx#1\count#2\fi
\ifx#1\dimen#2\fi
\ifx#1\skip#2\fi
\ifx#1\muskip#2\fi
\ifx#1\toks#2\fi
\ifx#1\box\setbox#2\fi
}%

% +--------------------------+
% | Misc Mathematical Macros |
% +--------------------------+

\def\m#1{\mskip#1mu}
\def\n#1{\mskip-#1mu}

% The \o and \c macros define numerically ordered open/close commands. These
% are easier than remembered the ordering of \bigl-\Biggl. The levels simply
% run from 0-4. Example: % \o2( x^2 \c2)

\def\o#1#2{%
\ifx#10#2\fi
\ifx#11\bigl#2\fi
\ifx#12\Bigl#2\fi
\ifx#13\biggl#2\fi
\ifx#14\Biggl#2\fi
}

\def\c#1#2{%
\ifx#10#2\fi
\ifx#11\bigr#2\fi
\ifx#12\Bigr#2\fi
\ifx#13\biggr#2\fi
\ifx#14\Biggr#2\fi
}

% Alignments often need some extra vertical breahting room.

\def\ccr{\cr\noalign{\vskip4pt}}

% Sometimes we need to cramp a formula to make it fit on a line. Example:
% \mathcramp0.8, \mathcramp0.3, etc...

\def\mathcramp#1{%
\thinmuskip=#1\thinmuskip
\medmuskip=#1\medmuskip
\thickmuskip=#1\thickmuskip
}%

% Make the "*" equivalent to \dot
\mathcode`\*="2201

\let\x=\times
\let\p=\partial
\let\del=\nabla

% Units of measure. Example: 10\s{in}

\def\s#1{\kern 1.75pt {\rm #1}} 

% I'm so \over it

\def\q#1#2{{{#1} \over {#2}}}%

% I replace Knuth's \{ in \cases with my own \lbrace because I redefine \{ to
% include a \left marker further down.

\def\cases#1{\left\lbrace\,\vcenter{\normalbaselines\m@th
    \ialign{$##\hfil$&\quad##\hfil\crcr#1\crcr}}\right.}%

% This is just like \matrix, but it sets \displaystyle on every cell

\def\dmatrix#1{\null\,\vcenter{\normalbaselines\m@th
    \ialign{%
      \hfil$\displaystyle ##$\hfil&&\quad\hfil$\displaystyle ##$\hfil\crcr
      \mathstrut\crcr\noalign{\kern-\baselineskip}
      #1\crcr\mathstrut\crcr\noalign{\kern-\baselineskip}}}\,}%

\def\ihat{{\mb{\hat \imath}}}%
\def\jhat{{\mb{\hat \jmath}}}%
\def\khat{{\mb{\hat k}}}%

\def\vlen#1{\left|\left|#1\right|\right|}%

\def\norm{\mathop{\rm norm}\nolimits}%
\def\sign{\mathop{\rm sign}\nolimits}%
\def\len{\mathop{\rm len}\nolimits}%
\def\abs{\mathop{\rm abs}\nolimits}%
\def\div{\mathop{\rm div}\nolimits}%
\def\grad{\mathop{\rm grad}\nolimits}%
\def\curl{\mathop{\rm curl}\nolimits}%
\def\arccosh{\mathop{\rm arccosh}\nolimits}%
\def\arcsinh{\mathop{\rm arcsinh}\nolimits}%
\def\ach{\mathop{\rm arcch}\nolimits}%
\def\ash{\mathop{\rm arcsh}\nolimits}%
\def\ch{\mathop{\rm ch}\nolimits}%
\def\sh{\mathop{\rm sh}\nolimits}%

% The d'Alembertian operator. How is this not core!?

\def\dal{{\raise1.0pt\hbox{${\XeTeXmathchar"1\xitsfam"29E0\m1}^2$}}}

% Use \cs a_x^2 instead of a_x^2. It's almost the same, but it looks slightly
% better to not have the subscript and superscript vertically aligned.

\def\cs#1_#2^#3{{#1}_{#2}^{\mskip3.5mu #3}}%

% Plain TeX's math skip macros are ,>; but I'm redefining them to be ,;: so
% that we can have \> for other purposes. Plus ,;: is a more logical
% progression anyway. Use \? for skips that would make the reader ask "is that
% even really there?"

\def\?{\mskip0.5\thinmuskip}%
\def\,{\mskip\thinmuskip}%
\def\;{\mskip\medmuskip}%
\def\:{\mskip\thickmuskip}%

% Negative skips, \!, \!; \!:, and \!?

\def\!#1{%
\ifnum`#1=`\?
\mskip-0.5\thinmuskip
\else\ifnum`#1=`\,
\mskip-\thinmuskip
\else\ifnum`#1=`\;
\mskip-\medmuskip
\else\ifnum`#1=`\:
\mskip-\thickmuskip
\fi\fi\fi\fi
}%

% Parenthesis, square braces, curly brackets, and chevrons. Plain TeX already
% defines some of these macros in slightly less useful ways, so we can't
% always use \prot here.

\def\({\left(}%
\def\){\right)}%

\def\[{\left[}%
\def\]{\right]}%

\def\{{\left\lbrace}%
\def\}{\right\rbrace}%

\def\<{\left<}%
\def\>{\right>}%

\def\ket#1{\left|\?\sv{#1}\?\right>}%
\def\bra#1{\left<\?\sv{#1}\?\right|}%
\def\braket#1#2{\left<\?\sv{#1}\?\middle|\?\sv{#2}\?\right>}%

% +-------------+
% | Misc Macros |
% +-------------+

\prot\def\dk#1{{\bf #1}}
\prot\def\ek#1{{\it #1}}
\prot\def\sk#1{{\sl #1}}

\prot\def\hs{\mskip40mu}
\prot\def\vs{\noalign{\vskip3pt}}

\prot\def\R#1{#1}

\def\unroll#1{\copy#1\vskip-\ht#1}%

% This macro comes from transforms.tex in the XeTeX docs

\def\rotate#1#2{{\setbox0=\hbox{#1}\rlap{\kern0.5\wd0
  \dimen0=\ht0 \advance\dimen0 by -\dp0
  \raise0.5\dimen0\hbox{\special{x:gsave}\special{x:rotate #2}}}%
  \box0 \special{x:grestore}}}%

% +---------------------+
% | Page number control |
% +---------------------+

\headline={\hss{\bf Page \folio}}%
\footline={\hfil}%
\def\nopagenumbers{\headline{\hfil}}%

\def\sv#1{{\mathbold=1\fam\bffam #1}}%

% These values are generated by taking 1.2^n * 1000 for n=0,1,2,...,9

\def\magstep#1{%
\ifcase#1
1000\or 1200\or 1440\or 1728\or 2074\or 2488\or 2986\or 3583\or 4300\or 5160%
\fi\relax
}%

% +---------------------+
% | Fonts & Font Charts |
% +---------------------+

% The plain.tex file suggests these three fonts for titles. In plain.tex
% cmtt10 is set to \magstep2, but this leaves it looking smaller than the
% other two. Giving cmtt10 a scale of 1625 instead of \magstep2 makes the
% X-heights approximately equal in all three fonts.

\font\titlefontrm=cmr7 scaled 2000
\font\titlefonttt=cmtt10 scaled 1400
\font\titlefontbf=cmssbx10 scaled 1400

% Plain TeX has the a-z and A-Z math characters in class 0. Having them in
% class 0 forces us to use the font in family 1 for every math letter in a
% given formula, and the font in family 1 is essentially always going to be
% cmmi{10,7,5}. However, we might want to dynamically mix math characters from
% various fonts, even in the same display. One method might be to reassign
% these characters to class 7 so that they can use the class 7 + \fam mapping
% mechanism. Unfortunately that breaks accents like \hat and \dot and such,
% because they rely on class 7 being perpetually mapped to something like
% cmr10, which has a very different character table compared to cmmi10.

\ivar\count\mathbold=0

\def\bm{\mathbold=1\fam\cmmibfam}%
\def\mb#1{{\bm #1}}%

% Math Character Classes (TeX Book Chapter 17, "More about Math")
% 0: Ordinary
% 1: Large operator
% 2: Binary operation
% 3: Relation
% 4: Opening
% 5: Closing
% 6: Punctuation
% 7: Variable family via \fam

% \@mathchar
% #1: The " character
% #2: Class
% #3: Family
% #4: Char
% #5: Char

\def\@mathchar#1#2#3#4#5{%
\ifnum#3=1
\ifnum\mathbold=1
\mathchar\numexpr"1000 * "#2 + "100 * \cmmibfam + "10 * "#4 + "#5\relax
\else
\mathchar\numexpr"1000 * "#2 + "100 * "#3 + "10 * "#4 + "#5\relax
\fi
\else
\ifnum\mathbold=1
\mathchar\numexpr"1000 * "#2 + "100 * \cmbsyfam + "10 * "#4 + "#5\relax
\else
\mathchar\numexpr"1000 * "#2 + "100 * "#3 + "10 * "#4 + "#5\relax
\fi
\fi
}%

% \@mathaccent
% #1: The " character
% #2: Class
% #3: Family
% #4: Char
% #5: Char

\def\@mathaccent#1#2#3#4#5{%
\ifnum\mathbold=1
\mathaccent\numexpr"1000 * "#2 + "100 * \bffam + "10 * "#4 + "#5\relax
\else
\mathaccent\numexpr"1000 * "#2 + "100 * "#3 + "10 * "#4 + "#5\relax
\fi
}%

\def\acute{\@mathaccent"0013 }%
\def\grave{\@mathaccent"0012 }%
\def\ddot{\@mathaccent"007F }%
\def\tilde{\@mathaccent"007E }%
\def\bar{\@mathaccent"0016 }%
\def\breve{\@mathaccent"0015 }%
\def\check{\@mathaccent"0014 }%
\def\hat{\@mathaccent"005E }%
\def\dot{\@mathaccent"005F }%
% Why is this one disabled?
%\def\vec{\@mathaccent"017E }%

\def\alpha{\@mathchar"010B}%
\def\beta{\@mathchar"010C}%
\def\gamma{\@mathchar"010D}%
\def\delta{\@mathchar"010E}%
\def\epsilon{\@mathchar"010F}%
\def\zeta{\@mathchar"0110}%
\def\eta{\@mathchar"0111}%
\def\theta{\@mathchar"0112}%
\def\iota{\@mathchar"0113}%
\def\kappa{\@mathchar"0114}%
\def\lambda{\@mathchar"0115}%
\def\mu{\@mathchar"0116}%
\def\nu{\@mathchar"0117}%
\def\xi{\@mathchar"0118}%
\def\pi{\@mathchar"0119}%
\def\rho{\@mathchar"011A}%
\def\sigma{\@mathchar"011B}%
\def\tau{\@mathchar"011C}%
\def\upsilon{\@mathchar"011D}%
\def\phi{\@mathchar"011E}%
\def\chi{\@mathchar"011F}%
\def\psi{\@mathchar"0120}%
\def\omega{\@mathchar"0121}%
\def\varepsilon{\@mathchar"0122}%
\def\vartheta{\@mathchar"0123}%
\def\varpi{\@mathchar"0124}%
\def\varrho{\@mathchar"0125}%
\def\varsigma{\@mathchar"0126}%
\def\varphi{\@mathchar"0127}%

\def\imath{\@mathchar"017B}%
\def\jmath{\@mathchar"017C}%

\newfam\cmmibfam
\newfam\cmbsyfam
\newfam\msbmfam
\newfam\msamfam
\newfam\esintfam
\newfam\lmmfam
\newfam\xitsfam
\newfam\asanafam

\def\bb{\fam\msbmfam\msbm}%
\font\msbm=msbm10
\font\tenmsbm=msbm10
\font\sevenmsbm=msbm7
\font\fivemsbm=msbm5
\textfont\msbmfam=\tenmsbm
\scriptfont\msbmfam=\sevenmsbm
\scriptscriptfont\msbmfam=\fivemsbm

\font\msam=msam10
\font\tenmsam=msam10
\font\sevenmsam=msam7
\font\fivemsam=msam5
\textfont\msamfam=\tenmsam
\scriptfont\msamfam=\sevenmsam
\scriptscriptfont\msamfam=\fivemsam

\font\esint=esint10
\textfont\esintfam=\esint
\scriptfont\esintfam=\esint
\scriptscriptfont\esintfam=\esint

\font\tencmmib=cmmib10
\font\sevencmmib=cmmib7
\font\fivecmmib=cmmib5
\skewchar\tencmmib='177
\skewchar\sevencmmib='177
\skewchar\fivecmmib='177
\textfont\cmmibfam=\tencmmib
\scriptfont\cmmibfam=\sevencmmib
\scriptscriptfont\cmmibfam=\fivecmmib

\font\tencmbsy=cmbsy10
\font\sevencmbsy=cmbsy7
\font\fivecmbsy=cmbsy5
\skewchar\tencmbsy='60
\skewchar\sevencmbsy='60
\skewchar\fivecmbsy='60
\textfont\cmbsyfam=\tencmbsy
\scriptfont\cmbsyfam=\sevencmbsy
\scriptscriptfont\cmbsyfam=\fivecmbsy

% Should this family have a skewchar?
\ifdefined\XeTeXversion
\font\tenxits="XITS Math" at 10pt
\font\sevenxits="XITS Math" at 7pt
\font\fivexits="XITS Math" at 5pt
\textfont\xitsfam=\tenxits
\scriptfont\xitsfam=\sevenxits
\scriptscriptfont\xitsfam=\fivexits
\fi

% Should this family have a skewchar?
\ifdefined\XeTeXversion
\font\tenlmm="Latin Modern Math" at 10pt
\font\sevenlmm="Latin Modern Math" at 7pt
\font\fivelmm="Latin Modern Math" at 5pt
\textfont\lmmfam=\tenlmm
\scriptfont\lmmfam=\sevenlmm
\scriptscriptfont\lmmfam=\fivelmm
\fi

% Should this family have a skewchar?
\ifdefined\XeTeXversion
\font\tenasana="Asana Math" at 10pt
\font\sevenasana="Asana Math" at 7pt
\font\fiveasana="Asana Math" at 5pt
\textfont\asanafam=\tenasana
\scriptfont\asanafam=\sevenasana
\scriptscriptfont\asanafam=\fiveasana
\fi

\ifdefined\XeTeXversion
\def\iint{\XeTeXmathchar"1\lmmfam"222C\nolimits}%
\def\iiint{\XeTeXmathchar"1\lmmfam"222D\nolimits}%
\def\oint{\XeTeXmathchar"1\lmmfam"222E\nolimits}%
\def\oiint{\XeTeXmathchar"1\lmmfam"222F\nolimits}%
\def\oiiint{\XeTeXmathchar"1\lmmfam"2230\nolimits}%
\def\ointcw{\XeTeXmathchar"1\lmmfam"2232\nolimits}%
\def\ointccw{\XeTeXmathchar"1\lmmfam"2233\nolimits}%
\else
\def\iint{\mathchar\numexpr"1003 + "100 * \esintfam\relax\nolimits}%
\def\iiint{\mathchar\numexpr"1005 + "100 * \esintfam\relax\nolimits}%
\def\oint{\mathchar\numexpr"100B + "100 * \esintfam\relax\nolimits}%
\def\oiint{\mathchar\numexpr"100D + "100 * \esintfam\relax\nolimits}%
\def\ointcw{\mathchar\numexpr"1017 + "100 * \esintfam\relax\nolimits}%
\def\ointccw{\mathchar\numexpr"1019 + "100 * \esintfam\relax\nolimits}%
\fi

% \fontchart
% #1: Font name
% #2: Chart label

\def\@fcb{{\fontchartfont\char\cref1}\global\advance\cref1 by 1\relax}%
\def\@frs{{\rm\number\cref0}\global\advance\cref0 by 1\relax}%
\def\fontchart#1#2{
\font\fontchartfont=#1
\pushc{0}%
\pushc{0}%
\vbox{
\hbox{\bf #2 [#1]}%
\kern6pt
\hrule
\vskip6pt
\halign{
\vrule height 8.5pt depth 9.5pt width 0pt
##\quad&
\hfil\enskip##\enskip\hfil&\hfil\enskip##\enskip\hfil&%
\hfil\enskip##\enskip\hfil&\hfil\enskip##\enskip\hfil&%
\hfil\enskip##\enskip\hfil&\hfil\enskip##\enskip\hfil&%
\hfil\enskip##\enskip\hfil&\hfil\enskip##\enskip\hfil&%
\hfil\enskip##\enskip\hfil&\hfil\enskip##\enskip\hfil&%
\hfil\enskip##\enskip\hfil&\hfil\enskip##\enskip\hfil&%
\hfil\enskip##\enskip\hfil&\hfil\enskip##\enskip\hfil&%
\hfil\enskip##\enskip\hfil&\hfil\enskip##\enskip\hfil\cr
&0&1&2&3&4&5&6&7&8&9&A&B&C&D&E&F\cr
\@frs&\@fcb&\@fcb&\@fcb&\@fcb&\@fcb&\@fcb&\@fcb&\@fcb&%
\@fcb&\@fcb&\@fcb&\@fcb&\@fcb&\@fcb&\@fcb&\@fcb\cr
\@frs&\@fcb&\@fcb&\@fcb&\@fcb&\@fcb&\@fcb&\@fcb&\@fcb&%
\@fcb&\@fcb&\@fcb&\@fcb&\@fcb&\@fcb&\@fcb&\@fcb\cr
\@frs&\@fcb&\@fcb&\@fcb&\@fcb&\@fcb&\@fcb&\@fcb&\@fcb&%
\@fcb&\@fcb&\@fcb&\@fcb&\@fcb&\@fcb&\@fcb&\@fcb\cr
\@frs&\@fcb&\@fcb&\@fcb&\@fcb&\@fcb&\@fcb&\@fcb&\@fcb&%
\@fcb&\@fcb&\@fcb&\@fcb&\@fcb&\@fcb&\@fcb&\@fcb\cr
\@frs&\@fcb&\@fcb&\@fcb&\@fcb&\@fcb&\@fcb&\@fcb&\@fcb&%
\@fcb&\@fcb&\@fcb&\@fcb&\@fcb&\@fcb&\@fcb&\@fcb\cr
\@frs&\@fcb&\@fcb&\@fcb&\@fcb&\@fcb&\@fcb&\@fcb&\@fcb&%
\@fcb&\@fcb&\@fcb&\@fcb&\@fcb&\@fcb&\@fcb&\@fcb\cr
\@frs&\@fcb&\@fcb&\@fcb&\@fcb&\@fcb&\@fcb&\@fcb&\@fcb&%
\@fcb&\@fcb&\@fcb&\@fcb&\@fcb&\@fcb&\@fcb&\@fcb\cr
\@frs&\@fcb&\@fcb&\@fcb&\@fcb&\@fcb&\@fcb&\@fcb&\@fcb&%
\@fcb&\@fcb&\@fcb&\@fcb&\@fcb&\@fcb&\@fcb&\@fcb\cr
\@frs&\@fcb&\@fcb&\@fcb&\@fcb&\@fcb&\@fcb&\@fcb&\@fcb&%
\@fcb&\@fcb&\@fcb&\@fcb&\@fcb&\@fcb&\@fcb&\@fcb\cr
\@frs&\@fcb&\@fcb&\@fcb&\@fcb&\@fcb&\@fcb&\@fcb&\@fcb&%
\@fcb&\@fcb&\@fcb&\@fcb&\@fcb&\@fcb&\@fcb&\@fcb\cr
\@frs&\@fcb&\@fcb&\@fcb&\@fcb&\@fcb&\@fcb&\@fcb&\@fcb&%
\@fcb&\@fcb&\@fcb&\@fcb&\@fcb&\@fcb&\@fcb&\@fcb\cr
\@frs&\@fcb&\@fcb&\@fcb&\@fcb&\@fcb&\@fcb&\@fcb&\@fcb&%
\@fcb&\@fcb&\@fcb&\@fcb&\@fcb&\@fcb&\@fcb&\@fcb\cr
\@frs&\@fcb&\@fcb&\@fcb&\@fcb&\@fcb&\@fcb&\@fcb&\@fcb&%
\@fcb&\@fcb&\@fcb&\@fcb&\@fcb&\@fcb&\@fcb&\@fcb\cr
\@frs&\@fcb&\@fcb&\@fcb&\@fcb&\@fcb&\@fcb&\@fcb&\@fcb&%
\@fcb&\@fcb&\@fcb&\@fcb&\@fcb&\@fcb&\@fcb&\@fcb\cr
\@frs&\@fcb&\@fcb&\@fcb&\@fcb&\@fcb&\@fcb&\@fcb&\@fcb&%
\@fcb&\@fcb&\@fcb&\@fcb&\@fcb&\@fcb&\@fcb&\@fcb\cr
\@frs&\@fcb&\@fcb&\@fcb&\@fcb&\@fcb&\@fcb&\@fcb&\@fcb&%
\@fcb&\@fcb&\@fcb&\@fcb&\@fcb&\@fcb&\@fcb&\@fcb\cr
}}}%

\def\cmcharts{
\fontchart{cmr10}{Roman}\vfill\eject
\fontchart{cmbx10}{Roman Bold}\vfill\eject
\fontchart{cmsl10}{Roman Slanted}\vfill\eject
\fontchart{cmss10}{Sans Serif}\vfill\eject
\fontchart{cmssbx10}{Sans Serif Bold}\vfill\eject
\fontchart{cmti10}{Italic}\vfill\eject
\fontchart{cmu10}{Italic Unslanted}\vfill\eject
\fontchart{cmtt10}{Typewriter}\vfill\eject
\fontchart{cmsltt10}{Typewriter Slanted}\vfill\eject
\fontchart{cmmi10}{Math Italic}\vfill\eject
\fontchart{cmmib10}{Math Italic Bold}\vfill\eject
\fontchart{cmsy10}{Math Symbols}\vfill\eject
\fontchart{cmbsy10}{Math Symbols Bold}\vfill\eject
\fontchart{cmex10}{Math Extensions}\vfill\eject
\fontchart{cmcsc10}{Caps and Small Caps}\vfill\eject
\fontchart{cmdunh10}{Dunhill Style}\vfill\eject
}%

\def\eulercharts{
\fontchart{eurm10}{Roman}\vfill\eject
\fontchart{eurb10}{Roman Bold}\vfill\eject
\fontchart{eusm10}{Script}\vfill\eject
\fontchart{eusb10}{Script Bold}\vfill\eject
\fontchart{eufm10}{Fraktur}\vfill\eject
\fontchart{eufb10}{Fraktur Bold}\vfill\eject
\fontchart{euex10}{Math Extensions}\vfill\eject
}%

\def\amscharts{
\fontchart{msam10}{AMS Symbols Set A}\vfill\eject
\fontchart{msbm10}{AMS Symbols Set B}\vfill\eject
}%

\def\fontcharts{%
\cmcharts
\eulercharts
\amscharts
}%

%%
%%
%%

\font\papertitlefont=cmbx12 scaled \numexpr 1400 * 10/12 \relax
\font\paperinfofont=cmr10 scaled 800 \relax
\font\paperinfofontsl=cmsl10 scaled 800 \relax
\let\papersectionfont=\bf

\prot\def\papertitle#1{%
\hbox{\papertitlefont #1}%
\vskip1pt
\hbox to \hsize{%
\paperinfofont
}
\hrule width \hsize
}

\def\paperheadingnoskip#1{%
\leaders\hbox{}\vskip18pt plus 6pt minus 6pt
\hbox{\papersectionfont [\hskip2pt#1\hskip2pt]}\nobreak
}

\def\paperheading#1{%
\paperheadingnoskip{#1}
\kern6pt\nobreak
}

% At the top of the file ``@'' is given a catcode of 11. This resets its
% catcode back to 12, which is its original value.

\catcode`\@=12

\input ../standard.tex
        
\papertitle{Polar Coordinates}

\paperheading{Polar Vectors}

Let $r(t)$ and $\theta(t)$ be functions of time. With these variables the
position of a particle can be represented as.
$$
\sv r(t)
=
r(t) * \cos(\theta(t)) \; \ihat +
r(t) * \sin(\theta(t)) \; \jhat
$$
For readability we can omit explicily writing the time parameter. With this
omission the possition can now be written as just
$$
\sv r
=
r \cos(\theta) \; \ihat +
r \sin(\theta) \; \jhat
$$
Sometimes we want to write a vector with the pair notation $(x, y)$ instead of
using the $\ihat$ and $\jhat$ basis vectors. In this notation our expression
becomes
$$
\sv r = (r \cos(\theta), r \sin(\theta))
$$
Which notation you use is mostly a matter of preference, but they can both be
more or less convenient at different times. We will work with both notations
in this paper so that we can become familiar with each of them. Let's start by
taking a derivative
$$
\q{d \sv r}{dt}
=
\q{d}{dt}\o2(r \cos(\theta) \; \ihat\c2) +
\q{d}{dt}\o2(r \sin(\theta) \; \jhat\c2)
$$
It looks like we now have to apply the product rule. Let's be extremly
explicit about it so that we don't miss anything:
$$
\eqalign{
\q{d \sv r}{dt}
=
&+ \q{dr}{dt} \cos(\theta) \, \ihat
+ r \q{d}{dt}\o2(\cos(\theta)\c2) \, \ihat
+ r \cos(\theta) \q{d\ihat}{dt}
\ccr
&+ \q{dr}{dt} \sin(\theta) \, \jhat
+ r \q{d}{dt}\o2(\sin(\theta)\c2) \, \jhat
+ r \sin(\theta) \q{d\jhat}{dt}
}
$$
Recall that the basis vectors $\ihat$ and $\jhat$ are constants, so $d\ihat/dt
= 0$ and $d\jhat/dt = 0$. When we drop these we get
$$
\q{d \sv r}{dt}
=
\q{dr}{dt} \cos(\theta) \, \ihat
+ r \q{d}{dt}\o2(\cos(\theta)\c2) \, \ihat
+ \q{dr}{dt} \sin(\theta) \, \jhat
+ r \q{d}{dt}\o2(\sin(\theta)\c2) \, \jhat
$$
Now let's take any derivatives involving $\cos$ and $\sin$. Recall that
$\theta$ is actually a function of time $\theta(t)$. The derivative of
$\sin(\theta)$ is actually the derivtive of $\sin(\theta(t))$, which requires
the use of the chain rule and comes to $\cos(\theta(t)) * d\theta/dt$.
$$
\q{d \sv r}{dt}
=
\q{dr}{dt} \cos(\theta) \, \ihat
- r \q{d\theta}{dt}\sin(\theta) \, \ihat
+ \q{dr}{dt} \sin(\theta) \, \jhat
+ r \q{d\theta}{dt}\cos(\theta) \, \jhat
$$
Now observe that we can make the following factorization
$$
\q{d \sv r}{dt}
=
\left.\q{dr}{dt}\middle(\cos(\theta) \, \ihat
+ \sin(\theta) \, \jhat\right)
+ \left.r \q{d\theta}{dt}\middle(-\sin(\theta) \, \ihat
+ \cos(\theta) \, \jhat\right)
$$
Observe furthermore that
$$
\q{1}{r}\sv r = \cos(\theta) \, \ihat + \sin(\theta) \, \jhat
$$
We can therefore write
$$
\q{d \sv r}{dt}
=
\q{dr}{dt}\q{1}{r}\sv r
+ \left.r \q{d\theta}{dt}\middle(-\sin(\theta) \, \ihat
+ \cos(\theta) \, \jhat\right)
$$
Now let's do some notational cleanup. We use Newton's notation to write $\dot
r = dr/dt$ and $\dot\theta = d\theta/dt$. We can also give the name $\sv v$ to
$d\sv r/dt$, since this is after all the veclotiy. Finally, the vector $\sv
r/r$ is conventionally given the name $\sv{\hat r}$. This vector points in the
direction of $\sv r$, but has unit length (meaning that it's length is $1$).
With all this new notation we now have
$$
\sv v
=
\dot r \? \sv{\hat r}
+ r \dot\theta\o1(-\sin(\theta) \, \ihat + \cos(\theta) \, \jhat \, \c1)
$$
Moreover, the derivative $\dot\theta$ is often given the even shorter name
$\omega$. This is called the angular velocity. Making this final replacement
gives us
$$
\sv v
=
\dot r \? \sv{\hat r}
+ r \omega\o1(-\sin(\theta) \, \ihat + \cos(\theta) \, \jhat \, \c1)
$$
This expression is now fairly compact and elegant, expect for the relatively
lengthy vector \hbox{$-\sin(\theta) \, \ihat + \cos(\theta) \, \jhat$}. I'll
leave it to you to study this vector, discover some of its interesting
properties, and perhaps give it a name of its own. It will also be instructive
to take the derivtive again, but this time using the notation $\sv r = (r
\cos(\theta), r \sin(\theta))$, with $\ihat$ and $\jhat$ left out completey.
Try doing this as an exercise as well.

\bye
